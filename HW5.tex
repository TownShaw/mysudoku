\documentclass[UTF8]{ctexart}
\usepackage[left=1.5cm,right=1.5cm,top=1.5cm,bottom=1.5cm]{geometry}
\usepackage{xcolor}
\usepackage{amsmath}
\usepackage{amssymb}
\usepackage{graphicx}
\usepackage{bm}
\usepackage[thmmarks,amsmath]{ntheorem}

\begin{document}
    \title{人工智能基础第五次作业}
    \author{肖桐 PB18000037}
    \date{2021 年 5 月 3 日}
    \maketitle

    \textnormal{\textbf{7.13.}}\newline
    (a). 由 implication elimination 有:
    $$
    (\alpha \Rightarrow \beta) \Leftrightarrow (\neg\alpha \vee \beta)
    $$
    因此有:
    $$
    \neg(P_1 \wedge P_2 \wedge \cdots \wedge P_m) \vee Q \Leftrightarrow ((P_1 \wedge P_2 \wedge \cdots \wedge P_m) \Rightarrow Q)
    $$
    由 $De Morgan$ 律, 有:
    $$
    \neg(P_1 \wedge P_2 \wedge \cdots \wedge P_m) \Leftrightarrow (\neg P_1 \vee \neg P_2 \vee \cdots \vee \neg P_m)
    $$
    因此有:
    $$
    (\neg P_1 \vee \neg P_2 \vee \cdots \vee \neg P_m \vee Q) \Leftrightarrow ((P_1 \wedge P_2 \wedge \cdots \wedge P_m) \Rightarrow Q)
    $$
    (b). 将 positive literals 记为 $Q_q \vee \cdots \vee Q_n$, negtive literals 记为 $\neg P_1 \vee \cdots \vee \neg P_m$.\newline
    于是该语句可以表示为:
    $$
    (\neg P_1 \vee \cdots \neg P_m) \vee (Q_q \vee \cdots \vee Q_n)
    $$
    由 (a) 中结论可得:
    $$
    (P_1 \wedge \cdots \wedge P_m) \Rightarrow (Q_q \vee \cdots \vee Q_n)
    $$
    (c). 

    \textnormal{\textbf{证明:}}
    
\end{document}